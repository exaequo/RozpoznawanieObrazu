% vim:encoding=utf8 ft=tex sts=2 sw=2 et:

\documentclass{classrepshort}
\usepackage[utf8]{inputenc}
\usepackage{color}
\usepackage{graphicx}
\usepackage{float}

\studycycle{Informatyka, studia dzienne, mgr II st.}
\coursesemester{II}

\coursename{Rozpoznawanie obrazów}
\courseyear{2016/2017}

\courseteacher{dr inż. Bartłomiej Stasiak}
\coursegroup{wtorek, 12:00}

\author{
  \studentinfo{Hubert Marcinkowski}{214942} \and
  \studentinfo{Artur Wróblewski}{214985}
}

\title{Zadanie 1}

\begin{document}
\maketitle

\section{MNIST - Wyniki}
\begin{table}[h!]
  \centering
  \caption{Wyniki jakości klasyfikacji oraz czasu obliczeń k-NN dla bazy MNIST dla różnych wartości $k$}
  \label{tab:tab1}
  \begin{tabular}{|c|c|c|}
    \hline
	k & jakość & czas[s]\\
    \hline
	1 & 75.89 & 3.024\\
    \hline
	3 & 78.26 & 7.410\\
	\hline
	5 & 79.59 & 11.446\\
	\hline
	7 & 80.25 & 15.924\\
	\hline
	9 & 80.73 & 19.927\\
	\hline
	11 & 80.47 & 24.083\\
	\hline
	13 & 80.67 & 28.342\\
	\hline
	15 & 80.58 & 32.804\\
	\hline
	19 & 80.55 & 42.011\\
	\hline
	51 & 79.968 & 111.047\\
	\hline
	53 & 79.60 & 113.578\\
	\hline
	99 & 78.80 & 113.852\\
	\hline
  \end{tabular}
\end{table}

\begin{table}[h!]
  \centering
  \caption{Macierz pomyłek k-NN dla bazy MNIST dla wszystkich cech oraz $k=9$}
  \label{tab:tab1}
  \begin{tabular}{|c|c|c|c|c|c|c|c|c|c|c|c|}
    \hline
	. & 0 & 1 & 2 & 3 & 4 & 5 & 6 & 7 & 8 & 9 & success ratio \\
    \hline
	0 & 932 & 0 & 13 & 5 & 2 & 3 & 1 & 0 & 22 & 2 & 95.10\\
    \hline
	1 & 0 & 1102 & 7 & 1 & 4 & 3 & 4 & 1 & 6 & 7 & 97.09\\
	\hline
	2 & 31 & 5 & 675 & 164 & 12 & 88 & 26 & 6 & 25 & 0 & 65.40\\
	\hline
	3 & 24 & 11 & 113 & 731 & 1 & 37 & 7 & 34 & 40 & 12 & 72.37\\
	\hline
	4 & 11 & 6 & 16 & 4 & 821 & 13 & 3 & 6 & 5 & 97 & 83.60\\
	\hline
	5 & 26 & 7 & 91 & 100 & 11 & 540 & 22 & 25 & 68 & 2 & 60.53\\
	\hline
	6 & 6 & 11 & 16 & 3 & 3 & 27 & 887 & 0 & 5 & 0 & 92.58\\
	\hline
	7 & 2 & 22 & 5 & 28 & 20 & 28 & 0 & 832 & 32 & 59 & 80.93\\
	\hline
	8 & 67 & 7 & 22 & 29 & 16 & 38 & 7 & 17 & 729 & 42 & 74.84\\
	\hline
	9 & 15 & 14 & 6 & 20 & 49 & 8 & 1 & 35 & 37 & 824 & 81.66\\    
    \hline
  \end{tabular}
\end{table}

\newpage

\section{STaR - Wyniki}

\begin{table}[h!]
  \centering
  \caption{Wyniki jakości klasyfikacji oraz czasu obliczeń k-NN dla bazy STaR dla różnych wartości $k$}
  \label{tab:tab1}
  \begin{tabular}{|c|c|c|}
    \hline
	k & jakość & czas[s]\\
    \hline
	1 & 74.00 & 13.170\\
    \hline
	3 & 74.00 & 13.009\\
	\hline
	5 & 72.00 & 12.822\\
	\hline
	7 & 73.33 & 13.080\\
	\hline
	9 & 71.33 & 12.987\\
	\hline
	11 & 70.66 & 13.118\\
	\hline
	13 & 71.33 & 13.089\\
	\hline
	15 & 70.66 & 12.831\\
	\hline
	31 & 60.66 & 13.224\\
	\hline
  \end{tabular}
\end{table}

\begin{table}[h!]
  \centering
  \caption{Macierz pomyłek k-NN dla całej bazy STaR dla wszystkich cech oraz $k=1$}
  \label{tab:tab1}
  \begin{tabular}{|c|c|c|c|c|c|c|c|c|c|c|c|}
    \hline
	. & 0 & 1 & 2 & 3 & 4 & 5 & 6 & 7 & 8 & 9 & success ratio \\
    \hline
	0 & 7 & 1 & 1 & 0 & 1 & 0 & 1 & 0 & 4 & 0 & 46.67\\
    \hline
	1 & 0 & 12 & 0 & 0 & 0 & 0 & 0 & 3 & 0 & 0 & 80.00\\
	\hline
	2 & 0 & 0 & 14 & 0 & 0 & 0 & 1 & 0 & 0 & 0 & 93.33\\
	\hline
	3 & 0 & 0 & 0 & 8 & 7 & 0 & 0 & 0 & 0 & 0 & 53.33\\
	\hline
	4 & 0 & 0 & 2 & 4 & 7 & 0 & 2 & 0 & 0 & 0 & 46.66\\
	\hline
	5 & 0 & 0 & 0 & 0 & 0 & 10 & 5 & 0 & 0 & 0 & 66.66\\
	\hline
	6 & 0 & 0 & 0 & 0 & 0 & 0 & 15 & 0 & 0 & 0 & 100.00\\
	\hline
	7 & 0 & 1 & 0 & 0 & 0 & 0 & 0 & 14 & 0 & 0 & 93.33\\
	\hline
	8 & 0 & 2 & 0 & 0 & 1 & 2 & 1 & 0 & 9 & 0 & 60.00\\
	\hline
	9 & 0 & 0 & 0 & 0 & 0 & 0 & 0 & 0 & 0 & 15 & 100.00\\    
    \hline
  \end{tabular}
\end{table} 


\begin{table}[h!]
  \centering
  \caption{Wyniki jakości klasyfikacji oraz czasu obliczeń k-NN dla poszczególnych reprezentacji elementów bazy STaR dla $k=1$}
  \label{tab:tab1}
  \begin{tabular}{|c|c|c|}
    \hline
	zbiór & jakość & czas[s]\\
    \hline
    test\_30st\_light & 50.00 & 8.694\\
    \hline
	test\_light & 82.00 & 9.063\\
	\hline
	test\_plain & 90.00 & 8.897\\
	\hline
  \end{tabular}
\end{table}


\end{document}
