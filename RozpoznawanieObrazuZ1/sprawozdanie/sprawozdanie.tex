% vim:encoding=utf8 ft=tex sts=2 sw=2 et:

\documentclass{classrep}
\usepackage[utf8]{inputenc}
\usepackage{color}
\usepackage{graphicx}
\usepackage{float}

\studycycle{Informatyka, studia dzienne, mgr II st.}
\coursesemester{II}

\coursename{Rozpoznawanie obrazów}
\courseyear{2016/2017}

\courseteacher{dr inż. Bartłomiej Stasiak}
\coursegroup{wtorek, 12:00}

\author{
  \studentinfo{Hubert Marcinkowski}{214942} \and
  \studentinfo{Artur Wróblewski}{214985}
}

\title{Zadanie 1}

\begin{document}
\maketitle


\section{Cel}
Zadanie polegało na stworzeniu szkieletu uniwersalnej aplikacji do rozpoznawania obiektów. W tym celu należało przygotować odpowiedni zestaw cech do klasyfikacji oraz opracować moduł jej dokonujący z wykorzystaniem zadanej metryki. W celu sprawdzenia stworzonej aplikacji należało wykorzystać ją do klasyfikacji obiektów 2 baz danych: MNIST oraz STaR.

\section{MNIST}

Baza MNIST zawiera ręcznie pisane cyfry. Składa się ze zbioru uczącego (60 000 przykładów) oraz testowego (10 000 przykładów). Każdy z przykładów to obraz pojedynczej cyfry.

\subsection{Zestaw cech}

Dla bazy MNIST zaproponowaliśmy użycie 8 cech:

\textsl{Uwaga: Każdy obraz traktowaliśmy jako binarny tj. piksel był uznawany za piksel wchodzący w skład obiektu, gdy jego jasność była większa niż 10 (w skali 0-255). Wszystkie inne uznawane są za tło.}

\textsl{Uwaga: Przy pierwszych 6 poniższych cechach dla każdej cyfry wyznaczaliśmy bryłę brzegową (ang. bounding box). Dzięki temu wyeliminowaliśmy przesunięcia cyfr w każdym kierunku.}

\begin{itemize}
\item \textbf{Ilość jasnych pikseli w dolnej połowie cyfry}\\
 Cyfry takie jak 6 czy 9 mają  różną ilość pikseli w górnej oraz dolnej połowie, dzięki czemu wraz z kolejną cechą można całkiem dobrze je odróżnić od np. 1, 8 czy 0.
\item \textbf{Ilość jasnych pikseli w górnej połowie cyfry}\\
 Analogicznie do poprzedniej cechy.
\item \textbf{Ilość ciemnych pikseli od lewej krawędzi cyfry do lewej krawędzi obrazu}\\
Sprawdzając każdy z wierszy zliczaliśmy ciemne piksele, aż do napotkania pierwszego piksela cyfry (jasnego). Połączenie tej oraz 4 kolejnych cech pozwoliło na rozpoznanie kształtu cyfr z każdej strony. Metoda ta nie jest jednak odporna na obroty cyfr oraz ich zmienną wysokość lub szerokość. Opis zliczania jest analogiczny dla 3 kolejnych cech.
\item \textbf{Ilość ciemnych pikseli od prawej krawędzi cyfry do prawej krawędzi obrazu}
\item \textbf{Ilość ciemnych pikseli od górnej krawędzi cyfry do górnej krawędzi obrazu}
\item \textbf{Ilość ciemnych pikseli od dolnej krawędzi cyfry do dolnej krawędzi obrazu}
\item \textbf{Odległość euklidesowa pikseli od środka cyfry}\\
 Środek wyznaczamy przy użyciu średniej arytmetycznej. Dzięki temu obliczyliśmy zwartość cyfry. Oczywiście część cyfr jest bardzo zbliżona pod tym względem.
\item \textbf{Stosunek wysokości do szerokości}\\
 Obliczyliśmy sumy odległości wszystkich pikseli od środków w dwóch kierunkach (od środka szerokości oraz środka wysokości). Ich stosunek pozwolił obliczyć "smukłość" cyfry. 
\end{itemize}

\subsection{Niewykorzystane cechy}

Początkowo próbowaliśmy użyć jeszcze kilku innych cech - odrzuciliśmy je jednak z powodu braku znacznego polepszenia wyników (a czasem nawet pogorszenia) oraz wydłużania czasu obliczeń.
\begin{itemize}
\item \textbf{Ilość jasnych pikseli cyfry}\\
 Statystycznie każda cyfra powinna mieć inną ilość pikseli składowych, niestety nie sprawdziło się to ze względu na różną wielkość cyfr.
\item \textbf{Pole bryły otaczającej}\\
 Celem było odróżnienie cyfr zajmujących mniejszą powierzchnię np. 1 oraz 0. Problemem był brak "odporności" na obroty cyfr.
\item \textbf{Stosunek ilości jasnych pikseli do pola powierzchni bryły otaczającej}
\item \textbf{Współczynniki kształtu oparte na momentach konturów}\\
Do wyznaczenia konturów użyliśmy filtracji liniowej.
\end{itemize}

\subsection{Wyniki}
\begin{table}[h!]
  \centering
  \caption{Wyniki jakości klasyfikacji oraz czasu obliczeń k-NN dla bazy MNIST dla różnych wartości $k$}
  \label{tab:tab1}
  \begin{tabular}{|c|c|c|}
    \hline
	k & jakość & czas[s]\\
    \hline
	1 & 73.67 & 69.783\\
    \hline
	3 & 76.08 & 67.983\\
	\hline
	5 & 77.90 & 68.175\\
	\hline
	7 & 78.01 & 68.076\\
	\hline
	9 & 78.31 & 67.515\\
	\hline
	11 & 78.58 & 68.414\\
	\hline
	13 & 78.26 & 67.970\\
	\hline
	15 & 77.94 & 67.725\\
	\hline
	19 & 78.00 & 67.416\\
	\hline
	35 & 77.54 & 67.814\\
	\hline
	99 & 75.70 & 68.476\\
	\hline
  \end{tabular}
\end{table}

\begin{table}[h!]
  \centering
  \caption{Wyniki jakości klasyfikacji k-NN dla bazy MNIST dla różnego zestawu cech}
  \label{tab:tab1}
  \begin{tabular}{|c|c|}
    \hline
	wybrane cechy & jakość\\
    \hline
	1 & 21.19\\
    \hline
	2 & 21.69\\
	\hline
	3 & 18.96\\
	\hline
	4 & 17.19\\
	\hline
	5 & 18.86\\
	\hline
	6 & 21.54\\
	\hline
	7 & 24.84\\
	\hline
	8 & 22.58\\
	\hline
	7,8 & 35.62\\
	\hline
	3,4 & 29.54\\
	\hline
	1,2,6,7,8 & 66.97\\
	\hline
	1,3,4,5,6 & 65.59\\
	\hline
	3,4,5,6,7 & 67.26\\
	\hline
	3,4,5,6,8 & 63.99\\
	\hline
  \end{tabular}
\end{table}

\begin{table}[h!]
  \centering
  \caption{Macierz pomyłek k-NN dla bazy MNIST dla wszystkich cech oraz $k=11$}
  \label{tab:tab1}
  \begin{tabular}{|c|c|c|c|c|c|c|c|c|c|c|c|}
    \hline
	. & 0 & 1 & 2 & 3 & 4 & 5 & 6 & 7 & 8 & 9 & success ratio \\
    \hline
	0 & 925 & 0 & 15 & 3 & 11 & 1 & 3 & 0 & 18 & 4 & 94.38\\
    \hline
	1 & 0 & 1086 & 8 & 7 & 4 & 5 & 4 & 0 & 16 & 5 & 95.68\\
	\hline
	2 & 32 & 11 & 633 & 186 & 20 & 78 & 29 & 15 & 25 & 3 & 61.33\\
	\hline
	3 & 25 & 16 & 103 & 719 & 1 & 33 & 7 & 45 & 45 & 16 & 71.18\\
	\hline
	4 & 20 & 13 & 27 & 2 & 772 & 18 & 8 & 6 & 10 & 106 & 78.81\\
	\hline
	5 & 19 & 11 & 80 & 102 & 13 & 506 & 30 & 33 & 87 & 11 & 56.72\\
	\hline
	6 & 6 & 8 & 21 & 3 & 6 & 23 & 885 & 0 & 6 & 0 & 92.38\\
	\hline
	7 & 2 & 36 & 10 & 24 & 15 & 25 & 0 & 818 & 26 & 72 & 79.57\\
	\hline
	8 & 87 & 5 & 18 & 31 & 23 & 39 & 9 & 13 & 701 & 48 & 71.97\\
	\hline
	9 & 18 & 14 & 4 & 18 & 47 & 12 & 3 & 41 & 39 & 813 & 80.57\\    
    \hline
  \end{tabular}
\end{table}

\section{STaR}

Jest to baza obrazów dziesięciu obiektów. Na zdjęciach występują 3 rodzaje tła, a pozycja, obrót oraz powiększenie obiektu są dobrane losowo.

\subsection{Zestaw cech}

Dla bazy STaR zdecydowaliśmy się skorzystać z tzw. momentów obiektu (obrazu), a dokładnie niezmienników przekształceń. Metoda ta pozwala na rozpoznawanie wzorów  niezależnie od pozycji, rozmiaru czy obrotu. Jako cechy użyliśmy każdego z niezmienników (łącznie 7 cech) - wzory podajemy poniżej:\\

$I_1 = \eta_{20} + \eta_{02}$

$I_2 = (\eta_{20} - \eta_{02})^2 + 4\eta_{11}^2$

$I_3 = (\eta_{30} - 3\eta_{12})^2 + (3\eta_{21} - \eta_{03})^2$

$ I_4 = (\eta_{30} + \eta_{12})^2 + (\eta_{21} + \eta_{03})^2$

$ I_5 = (\eta_{30} - 3\eta_{12}) (\eta_{30} + \eta_{12})[ (\eta_{30} + \eta_{12})^2 - 3 (\eta_{21} + \eta_{03})^2] + (3 \eta_{21} - \eta_{03}) (\eta_{21} + \eta_{03})[ 3(\eta_{30} + \eta_{12})^2 -  (\eta_{21} + \eta_{03})^2]$

$I_6 =  (\eta_{20} - \eta_{02})[(\eta_{30} + \eta_{12})^2 - (\eta_{21} + \eta_{03})^2] + 4\eta_{11}(\eta_{30} + \eta_{12})(\eta_{21} + \eta_{03})$

$I_7 = (3 \eta_{21} - \eta_{03})(\eta_{30} + \eta_{12})[(\eta_{30} + \eta_{12})^2 - 3(\eta_{21} + \eta_{03})^2] - (\eta_{30} - 3\eta_{12})(\eta_{21} + \eta_{03})[3(\eta_{30} + \eta_{12})^2 - (\eta_{21} + \eta_{03})^2]$

\subsection{Niewykorzystane cechy}

Początkowo planowaliśmy wykorzystać cechy stworzone dla bazy MNIST, jednak żadna z nich nie była "odporna" na przesunięcie, obrót lub skalowanie.

\subsection{Wyniki}

\begin{table}[h!]
  \centering
  \caption{Wyniki jakości klasyfikacji oraz czasu obliczeń k-NN dla bazy STaR dla różnych wartości $k$}
  \label{tab:tab1}
  \begin{tabular}{|c|c|c|}
    \hline
	k & jakość & czas[s]\\
    \hline
	1 & 24.66 & 10.549\\
    \hline
	3 & 28.66 & 4.333\\
	\hline
	5 & 30.66 & 4.248\\
	\hline
	7 & 32.00 & 4.247\\
	\hline
	9 & 34.00 & 4.274\\
	\hline
	11 & 34.66 & 4.338\\
	\hline
	13 & 32.66 & 4.431\\
	\hline
	15 & 29.33 & 4.257\\
	\hline
	31 & 29.33 & 4.279\\
	\hline
	99 & 24.00 & 4.248\\
	\hline
  \end{tabular}
\end{table}

\begin{table}[h!]
  \centering
  \caption{Macierz pomyłek k-NN dla bazy STaR dla wszystkich cech oraz $k=11$}
  \label{tab:tab1}
  \begin{tabular}{|c|c|c|c|c|c|c|c|c|c|c|c|}
    \hline
	. & 0 & 1 & 2 & 3 & 4 & 5 & 6 & 7 & 8 & 9 & success ratio \\
    \hline
	0 & 4 & 5 & 0 & 0 & 1 & 1 & 1 & 0 & 0 & 3 & 26.67\\
    \hline
	1 & 1 & 8 & 0 & 1 & 0 & 0 & 0 & 0 & 0 & 5 & 53.33\\
	\hline
	2 & 0 & 3 & 3 & 2 & 0 & 1 & 0 & 0 & 0 & 6 & 20.00\\
	\hline
	3 & 1 & 1 & 0 & 8 & 2 & 2 & 0 & 0 & 0 & 1 & 53.33\\
	\hline
	4 & 3 & 4 & 0 & 7 & 1 & 0 & 0 & 0 & 0 & 0 & 6.66\\
	\hline
	5 & 0 & 1 & 2 & 0 & 0 & 7 & 3 & 0 & 2 & 0 & 46.66\\
	\hline
	6 & 1 & 2 & 2 & 0 & 0 & 2 & 8 & 0 & 0 & 0 & 53.33\\
	\hline
	7 & 1 & 5 & 2 & 2 & 0 & 1 & 1 & 0 & 0 & 3 & 0.00\\
	\hline
	8 & 1 & 3 & 0 & 4 & 1 & 1 & 3 & 0 & 0 & 2 & 0.00\\
	\hline
	9 & 0 & 0 & 0 & 0 & 0 & 1 & 0 & 0 & 1 & 13 & 86.66\\    
    \hline
  \end{tabular}
\end{table}

\section{Wnioski}

Wyniki dla bazy MNIST przy użyciu jedynie 8 cech są zadowalające. Dodatkowo czasy przetwarzania 10 000 elementów są relatywnie krótkie. Parametr $k$ nie wpływa w naszej implementacji na czas wykonania obliczeń - jedynie na jakość klasyfikacji. Tutaj warto zauważyć, że wraz ze wzrostem $k$ klasyfikator zwracał większa liczbę poprawnych wyników, aczkolwiek wynik najlepszy osiągnęliśmy przy $k=11$: $78.58\%$. Zwiększając coraz bardziej $k$ wynik staje się tylko gorszy. Wybór oraz zdefiniowanie odpowiedniego zestawu cech jest kluczowy przy tym rodzaju klasyfikacji. Mając jednak ich w tym przypadku 8 bardzo ciężko jest wybrać te, które wpłyną na osiągnięcie najlepszego wyniku. Postanowiliśmy sprawdzić to porównując różne zestawy: złożone z najlepszych statystycznie cech oraz tych najgorszych. Możemy powiedzieć, że jeśli istnieje jakiś związek między cechami, a końcową jakością klasyfikacji to będzie to raczej ilość użytych cech, niż fakt użycia najlepszych statystycznie (tu: uzyskujących najlepsze wyniki przy skorzystaniu tylko z jednej cechy). Analizując macierz pomyłek możemy zauważyć, że klasyfikator najlepiej sobie poradził z cyframi 1, 0 i 6 uzyskując wynik ponad $92\%$ dla każdej. Powyżej $70\%$ były kolejno 9, 7, 4, 8 oraz 3. Najgorzej wypadło rozpoznawanie cyfr 2 i 5 (poniżej $65\%$). Warto zauważyć, że najczęściej błędnie były klasyfikowane jako cyfra 3. Możliwe, że przez podobieństwo górnej połowy (do cyfry 2) oraz dolnej (do cyfry 5).

\end{document}
