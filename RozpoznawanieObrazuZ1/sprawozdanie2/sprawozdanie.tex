% vim:encoding=utf8 ft=tex sts=2 sw=2 et:

\documentclass{classrep}
\usepackage[utf8]{inputenc}
\usepackage{color}
\usepackage{graphicx}
\usepackage{float}

\studycycle{Informatyka, studia dzienne, mgr II st.}
\coursesemester{II}

\coursename{Rozpoznawanie obrazów}
\courseyear{2017/2018}

\courseteacher{dr inż. Bartłomiej Stasiak}
\coursegroup{wtorek, 12:00}

\author{
  \studentinfo{Hubert Marcinkowski}{214942} \and
  \studentinfo{Artur Wróblewski}{214985}
}

\title{Zadanie 2}

\begin{document}
\maketitle


\section{Cel}
Zadanie polegało na implementacji dodatkowej metody klasyfikacji w istniejącym już szkielecie z zadania nr 1. Dodatkowo należało dokonać analizy zdolności klasyfikacji obu metod wykorzystując obrazy z różnymi teksturami. Konieczne było również wprowadzenie dwóch metod ekstrakcji cech: w dziedzinie czasu oraz w dziedzinie częstotliwości.

\section{Własna metoda}
Zaimplementowaliśmy autorską metodę Marcinkovsky-Vroblevsky polegającą na wyznaczeniu centroidów każdej z klas oraz obliczeniu odległości od klasyfikowanego obiektu. Klasa do której należy nowy obiekt to ta do którego centrum odległość jest najmniejsza.

\section{Zestawy cech}

\subsection{Dziedzina częstotliwości}

Dla dziedziny częstotliwości zaproponowaliśmy użycie 4 cech. Każda z nich bazowała na sumie jasności pikseli w widmie amplitudowym z maską w kształcie pierścienia o promieniach (wewnętrzny i zewnętrzny):
\begin{itemize}
\item 2$px$ i 4$px$
\item 8$px$ i 10$px$
\item 14$px$ i 16$px$
\item 25$px$ i 27$px$
\end{itemize}

Były to dobrze nam znane filtry pasmowoprzepustowe. Wybierając promienie pierścieni kierowaliśmy się występowaniem najjaśniejszych składowych widma. Cechy te są niezależne od obrotu.

\subsection{Dziedzina czasu}

Dla dziedziny czasu zaproponowaliśmy użycie 8 cech:

\begin{itemize}
\item suma jasności pikseli po wykryciu krawędzi z wykorzystaniem operatora Laplace'a
\item 7 momentów obrazu przedstawionych poniżej
\end{itemize}

Momenty obiektu (obrazu), a dokładnie niezmienniki przekształceń. Metoda ta pozwala na rozpoznawanie wzorów  niezależnie od pozycji, rozmiaru czy obrotu. Jako cechy użyliśmy każdego z niezmienników (łącznie 7 cech) - wzory podajemy poniżej:\\

$I_1 = \eta_{20} + \eta_{02}$

$I_2 = (\eta_{20} - \eta_{02})^2 + 4\eta_{11}^2$

$I_3 = (\eta_{30} - 3\eta_{12})^2 + (3\eta_{21} - \eta_{03})^2$

$ I_4 = (\eta_{30} + \eta_{12})^2 + (\eta_{21} + \eta_{03})^2$

$ I_5 = (\eta_{30} - 3\eta_{12}) (\eta_{30} + \eta_{12})[ (\eta_{30} + \eta_{12})^2 - 3 (\eta_{21} + \eta_{03})^2] + (3 \eta_{21} - \eta_{03}) (\eta_{21} + \eta_{03})[ 3(\eta_{30} + \eta_{12})^2 -  (\eta_{21} + \eta_{03})^2]$

$I_6 =  (\eta_{20} - \eta_{02})[(\eta_{30} + \eta_{12})^2 - (\eta_{21} + \eta_{03})^2] + 4\eta_{11}(\eta_{30} + \eta_{12})(\eta_{21} + \eta_{03})$

$I_7 = (3 \eta_{21} - \eta_{03})(\eta_{30} + \eta_{12})[(\eta_{30} + \eta_{12})^2 - 3(\eta_{21} + \eta_{03})^2] - (\eta_{30} - 3\eta_{12})(\eta_{21} + \eta_{03})[3(\eta_{30} + \eta_{12})^2 - (\eta_{21} + \eta_{03})^2]$

\section{Wyniki}

\begin{table}[h!]
  \centering
  \caption{Macierz pomyłek k-NN dla bazy tekstur dla cech w dziedzinie czasu oraz $k=11$}
  \label{tab:tab1}
  \begin{tabular}{|c|c|c|c|c|c|}
    \hline
	. & 0 & 1 & 2 & 3 & success ratio \\
    \hline
	0 & 156927 & 15719 & 7351 & 4289 & 85.15\\
    \hline
	1 & 27461 & 173534 & 3193 & 388 & 84.82\\
	\hline
	2 & 55492 & 21346 & 87646 & 34788 & 43.98\\
	\hline
	3 & 66297 & 31129 & 28146 & 76951 & 37.99\\   
    \hline
  \end{tabular}
\end{table}

\begin{table}[h!]
  \centering
  \caption{Macierz pomyłek k-NN dla bazy tekstur dla cech w dziedzinie częstotliwości oraz $k=11$}
  \label{tab:tab1}
  \begin{tabular}{|c|c|c|c|c|c|}
    \hline
	. & 0 & 1 & 2 & 3 & success ratio \\
    \hline
	0 & 142291 & 11 & 35037 & 6947 & 77.21\\
    \hline
	1 & 153 & 144185 & 59780 & 458 & 70.47\\
	\hline
	2 & 1682 & 19914 & 171309 & 2142 & 87.82\\
	\hline
	3 & 24578 & 2162 & 64818 & 110965 & 54.79\\   
    \hline
  \end{tabular}
\end{table}

\begin{table}[h!]
  \centering
  \caption{Macierz pomyłek k-NN dla bazy tekstur dla cech w dziedzinie czasu oraz częstotliwości oraz $k=11$}
  \label{tab:tab1}
  \begin{tabular}{|c|c|c|c|c|c|}
    \hline
	. & 0 & 1 & 2 & 3 & success ratio \\
    \hline
	0 & 161173 & 820 & 13306 & 8987 & 87.45\\
    \hline
	1 & 180 & 176083 & 27727 & 586 & 86.07\\
	\hline
	2 & 4087 & 20949 & 172832 & 1404 & 86.73\\
	\hline
	3 & 34049 & 3172 & 47383 & 117919 & 58.22\\   
    \hline
  \end{tabular}
\end{table}

\begin{table}[h!]
  \centering
  \caption{Macierz pomyłek M-W dla bazy tekstur dla cech w dziedzinie czasu}
  \label{tab:tab1}
  \begin{tabular}{|c|c|c|c|c|c|}
    \hline
 . & 0 & 1 & 2 & 3 & success ratio \\
    \hline
 0 & 1112 & 176046 & 7128 & 0 & 6.12\\
    \hline
 1 & 13034 & 180124 & 4807 & 6611 & 88.05\\
 \hline
 2 & 365495 & 37775 & 95981 & 21 & 48.17\\
 \hline
 3 & 58600 & 75239 & 30708 & 37976 & 18.75\\   
    \hline
  \end{tabular}
\end{table}

\begin{table}[h!]
  \centering
  \caption{Macierz pomyłek M-W dla bazy tekstur dla cech w dziedzinie częstotliwości}
  \label{tab:tab1}
  \begin{tabular}{|c|c|c|c|c|c|}
    \hline
 . & 0 & 1 & 2 & 3 & success ratio \\
    \hline
 0 & 119005 & 675 & 64606 & 0 & 64.58\\
    \hline
 1 & 221 & 82197 & 122158 & 0 & 40.18\\
 \hline
 2 & 367 & 84615 & 110130 & 4160 & 55.27\\
 \hline
 3 & 74712 & 12033 & 54018 & 61760 & 30.75\\   
    \hline
  \end{tabular}
\end{table}

\begin{table}[h!]
  \centering
  \caption{Macierz pomyłek M-W dla bazy tekstur dla cech w dziedzinie czasu oraz częstotliwości}
  \label{tab:tab1}
  \begin{tabular}{|c|c|c|c|c|c|}
    \hline
 . & 0 & 1 & 2 & 3 & success ratio \\
    \hline
 0 & 119665 & 43814 & 20807 & 0 & 64.93\\
    \hline
 1 & 154 & 113042 & 91380 & 0 & 55.26\\
 \hline
 2 & 364 & 90841 & 98595 & 9472 & 49.48\\
 \hline
 3 & 64842 & 29359 & 46306 & 62016 & 30.62\\   
    \hline
  \end{tabular}
\end{table}


\section{Wnioski}

Dla zaimplementowanej przez nas metody zaobserwować można niestety wyraźny spadek jakości wyników względem metody k-NN. Dla przypadku brania pod uwagę jedynie cech z dziedziny czasu dla lnu zauważyć można rekordowo niską rozpoznawalność (stanowczo poniżej tej zadanej funkcją losującą). Mimo wszystko, sumaryczne wyniki dla każdego przypadku (dla atrybutów z dziedziny czasu, częstotliwości oraz połączonych) prezentują wyniki lepsze niż te dla rozkładu losowego (około 25\% dla każdej z 4 klas). Zauważyć można jednak, że reprezentacja za pomocą centroidów nie jest jednak wystarczająca do otrzymania przyciągających uwagę wyników klasyfikacji. Niewątpliwą zaletą autorskiej metody jest czas wykonania, który jest zauważalnie krótszy.

\end{document}
